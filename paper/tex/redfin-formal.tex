
This section presents the formal verification framework developed on top of
the REDFIN semantic core (\S\ref{sec-transformer}) demonstrating the following
steps of the workflow:


\begin{itemize}
    \item Develop programs either in low-level REDFIN assembly, or in a high-level
    and statically type-checked expression language embedded in Haskell.
    \item Initialise, execute and test REDFIN programs on concrete input values.
    \item Formulate and refine functional correctness and worst case execution time
    properties in the SBV property specification language.
    \item Verify the properties with an SMT solver.
    \item Receive the verification results (e.g. counterexamples) for analysis.
\end{itemize}

\noindent
As our running example, consider the following simple spacecraft control task.


\begin{tcolorbox}

Let $t_1$ and $t_2$ be two different time points (measured in ms),
and $p_1$ and $p_2$ be two power values (measured in mW).
Calculate the estimate of the total energy consumption during this period
using linear approximation, rounding down to the nearest integer:

\[
\textit{energyEstimate}(t_1, t_2, p_1, p_2) = \left\lfloor \frac{|t_1 - t_2| * (p_1 + p_2)}{2} \right\rfloor.
\]

\end{tcolorbox}

\noindent
This task looks too simple, but in fact it has a few critical pitfalls that,
if left unattended, may lead to the failure of the whole space mission. Examples
of subtle bugs in seemingly simple programs leading to a catastrophe include 64"/bit
to 16"/bit number conversion overflow causing the destruction of Ariane~5
rocket~\cite{bug-rocket} and the loss of NASA's Mars climate orbiter due to incorrect
units of measurement conversion~\cite{NASA:1999:Mars}. Let us develop and verify
a REDFIN program for this task.


\subsubsection{Writing the program}
We can write programs either directly in the untyped REDFIN assembly, or in a
typed higher-level expression language. Both have their advantages: the former
allows engineers to hand-craft highly optimised programs under tight resource
constraints, and the latter brings type-safety and faster prototyping. Our first
prototype therefore uses the high-level approach.

Using Haskell's polymorphism, we can define an expression that can
be used both as a Haskell function and a high-level REDFIN expression:


\begin{minted}{haskell}
energyEstimate :: Integral a
               => a -> a -> a -> a -> a
energyEstimate t1 t2 p1 p2 =
    abs (t1 - t2) * (p1 + p2) `div` 2
\end{minted}


\noindent
We implement the energy estimation program by embedding the~\hs{energyEstimate}
expression into a REDFIN~\hs{Script}. Due to the lack of space we omit the
implementation of \hs{Script}, but one can think of it as a restricted version
of the \hs{Redfin} state transformer, which we use to write \emph{programs that
can manipulate the processor state only by executing instructions}, e.g. the
only way to set the \hs{Overflow} flag is to execute an arithmetic instruction
that actually has an overflow.


\begin{minted}[linenos]{haskell}
energyEstimateHighLevel :: Script
energyEstimateHighLevel = do
    let t1    = read (IntegerVariable 0)
        t2    = read (IntegerVariable 1)
        p1    = read (IntegerVariable 2)
        p2    = read (IntegerVariable 3)
        temp  = Temporary 4
        stack = Stack 5
    compile r0 stack temp (energyEstimate t1 t2 p1 p2)
    halt
\end{minted}
\label{energyEstimateHighLevel}


\noindent
Here the type \hs{IntegerVariable} is used to statically distinguish between integer
and fixed-point numbers, \hs{Temporary} to mark temporary words, so they cannot
be mixed with inputs and outputs, and \hs{Stack} to denote the location of the
stack pointer. The~\hs{let} block declares six adjacent memory addresses: four
input values $\{t_1, t_2, p_1, p_2\}$, a temporary word and a stack pointer.
The~\hs{compile} function call at line~9 performs the embedding of the high-level
\hs{energyEstimate} expression into the assembly language by translating it to a
sequence of REDFIN instructions. The first argument of the~\hs{compile} function
holds the register~\hs{r0} which contains the estimated energy value after the
program execution.


\subsubsection{Simulating the program}
We run symbolic simulation for 100 steps, initialising the program and data
memory of the processor using the helper function \hs{boot}.

\begin{minted}{haskell}
main :: IO ()
main = do
    let dataMemory = [10, 5, 3, 5, 0, 100]
        finalState = simulate 100
            (boot energyEstimateHighLevel dataMemory)
    printMemoryDump 0 5 (memory finalState)
    putStrLn $ "R0: " ++
               show (readArray (registers finalState) r0)
\end{minted}

\noindent
As the simulation result we get a \hs{finalState} value. We can inspect it by
printing relevant components: the values of the first six memory cells, and the
result of the computation located in the register~\hs{r0}. Note that the stack
pointer (cell 5) holds 100 as in the initial state, which means the stack is empty.


\begin{minted}[frame=single, fontsize=\small]{text}
Memory dump: [10, 5, 3, 5, 5, 100]
R0: 20
\end{minted}


Simulating programs with concrete input values is useful for diagnostic and test.
However, to formally verify their functional correctness, we need to inspect
every valid combination of input values, which can be done by an SMT solver.
This allows us to discover a trap hidden in our energy estimation program.


\subsubsection{Verifying the program}
The project lead engineer defined a set of functional requirements for the
energy monitoring subsystem. The software engineering team received the
specification, implemented the energy monitoring subroutines, and started the
verification. One of the requirements is as follows.

\begin{tcolorbox}

Assuming that values $p_1$ and $p_2$ are non"/negative integers, the energy
estimation subroutine must always return a non-negative integer value.

\end{tcolorbox}

To check that the program complies with the requirement, we translate the
symbolic state transformer \hs{energyEstimateHighLevel} into an SMT formula,
and formulate the following theorem. Lines 13-14 extract the computed result and
the value of the flag \hs{Halt} from the \hs{finalState}. Lines 15-16 require that
the processor has halted, the result is equal to that computed by the high-level
Haskell expression \hs{energyEstimate} and is non"/negative.


\begin{minted}[linenos, fontsize=\small]{haskell}
theorem = do
    -- Initialise symbolic variables:
    t1 <- forall "t1"
    t2 <- forall "t2"
    p1 <- forall "p1"
    p2 <- forall "p2"
    -- Constrain p1 and p2 to be non-negative:
    constrain $ p1 .>= 0 &&& p2 .>= 0
    -- Initialise the data memory with symbolic variables:
    let dataMemory = [t1, t2, p1, p2, 0, 100]
        finalState = simulate 100
                     (boot energyEstimateHighLevel dataMemory)
        result     = readArray (registers finalState) r0
        halted     = readArray (flags finalState) (flagId Halt)
    -- Specify the requirements as a property for verification:
    return $ halted &&& result .== energyEstimate t1 t2 p1 p2
                    &&& result .>= 0
\end{minted}

\noindent
The resulting SMT formula has 122 clauses and can be checked by Z3 in
3.0s\footnote{All reported results are obtained on a laptop with 2.90GHz Intel
Core i5-4300U processor, 8GB RAM (3MB cache), and the SMT solver Z3 version 4.5.1 (64-bit).}:


\begin{minted}[frame=single]{text}
> proveWith z3 theorem
Falsifiable. Counter-example:
  t1 = 5190405167614263295 :: Int64
  t2 =                   0 :: Int64
  p1 =  149927859193384455 :: Int64
  p2 =  157447350457463356 :: Int64
\end{minted}


%TODO: Can we obtain a simpler counterexample using SBV optimisation?
The solver has found a counterexample demonstrating that the program does not
satisfy the above requirement. Indeed, the expression evaluates to a negative
value on the provided inputs due to an \emph{integer overflow}. The following
refined requirement is then provided by the project lead engineer:


\begin{tcolorbox}

According to the spacecraft power system specification, $p_1$ and $p_2$ are
non"/negative integers that do not exceed 1W. The time is measured
from the mission start, hence $t_1$ and $t_2$ are non-negative and do not exceed
the time span of the mission, which is 30 years. Under these assumptions,
the energy estimation subroutine must return a non-negative integer value.

\end{tcolorbox}

\noindent
The software engineering team modifies the time\footnote{We are not absolutely
precise here. We do not distinguish between regular and leap years, and use
a conservative upper bound of 366 days per year.} and power constraints:


\begin{minted}{haskell}
constrain $ t1 .>= 0
        &&& t1 .<= toMilliSeconds (30 :: Year)
constrain $ t2 .>= 0
        &&& t2 .<= toMilliSeconds (30 :: Year)
constrain $ p1 .>= 0
        &&& p1 .<= toMilliWatts   ( 1 :: Watt)
constrain $ p2 .>= 0
        &&& p2 .<= toMilliWatts   ( 1 :: Watt)
\end{minted}


\noindent
We rerun Z3 (now on 134 SMT clauses) and get the desired outcome in 4.8s:


\begin{minted}[frame=single]{text}
> proveWith z3 theorem
Q.E.D.
\end{minted}


\noindent
The refinement of the requirement has rendered the integer overflow impossible,
in particular, we can now be sure that \hs{abs} cannot be called with~$-2^{63}$
within the mission parameters. This kind of guarantee fundamentally requires
solving an SMT problem, even if it is done at the type level, e.g. using
\emph{refinement types}~\cite{vazou2014refinement}.


\subsubsection{Checking program equivalence}
The statically typed high-level expression language is very convenient for
writing REDFIN programs, however, an experienced engineer can often find a way
to improve the resulting code. In some particularly resource-constrained situations,
a fully hand-crafted assembly code may be required. As an example, a direct
unoptimised translation of the \hs{energyEstimate} expression into assembly uses
79 instructions, most of them for \hs{Stack} manipulation.
On the other hand, it is not difficult to write a low-level assembly program that
computes the result using only 9 instructions as we demonstrate below.

To facilitate the development of verified hand"/crafted code, we use an SMT
solver to check the equivalence of REDFIN programs by verifying that they produce
the same output on all valid inputs. This allows an engineer to manually
optimise a given high-level prototype and have a guarantee that no bugs were
introduced in the process.

An optimised low-level energy estimation program has only 9 instructions:


\begin{minted}{haskell}
energyEstimateLowLevel :: Script
energyEstimateLowLevel = do
    let { t1 = 0; t2 = 1; p1 = 2; p2 = 3 }
    ld r0 t1
    sub r0 t2
    abs r0
    ld r1 p1
    add r1 p2
    st r1 p2
    mul r0 p2
    sra_i r0 1
    halt
\end{minted}
\label{energyEstimateLowLevel}


\noindent
Below we define the \hs{equivalence} check of this low"/level program with the
high-level program \hs{energyEstimateHighLevel} introduced earlier.


\begin{minted}{haskell}
equivalence = do
    t1 <- forall "t1"
    t2 <- forall "t2"
    p1 <- forall "p1"
    p2 <- forall "p2"
    constrain $ t1 .>= 0
            &&& t1 .<= toMilliSeconds (30 :: Year)
    constrain $ t2 .>= 0
            &&& t2 .<= toMilliSeconds (30 :: Year)
    constrain $ p1 .>= 0
            &&& p1 .<= toMilliWatts   ( 1 :: Watt)
    constrain $ p2 .>= 0
            &&& p2 .<= toMilliWatts   ( 1 :: Watt)
    let dataMemory   = [t1, t2, p1, p2, 0, 100]
        llFinalState = simulate 100 (boot energyEstimateLowLevel  dataMemory)
        hlFinalState = simulate 100 (boot energyEstimateHighLevel dataMemory)
        llResult     = readArray (registers llFinalState) r0
        hlResult     = readArray (registers hlFinalState) r0
    return $ llResult .== hlResult
\end{minted}


\noindent
We run Z3 (now on 52 SMT clauses) and get the affirmative result in 11.5s:


\begin{minted}[frame=single]{text}
> proveWith z3 equivalence
Q.E.D.
\end{minted}



\subsubsection{Program timing analysis}

The REDFIN semantic core implements the system clock tracking with the~\hs{delay}
function:


\begin{minted}{haskell}
delay :: Clock -> Redfin ()
delay cycles =
    Redfin $ \s -> ((), s { clock = clock s + cycles })
\end{minted}

%TODO: Same can be done about power
Every execution step~-- a call of the~\hs{executeInstruction} function~-- advances
the clock by the number of cycles required to do the associated computation as well
as memory and register accesses. The tracking is precisely matched to the hardware
implementation and enables the engineers to
perform~\emph{best/worst case execution timing analysis} exploiting the
optimisation facilities provided by SBV and Z3. As an example, let us determine
the minimum and maximum number of clock cycles required for
executing~\hs{energyEstimateLowLevel}. For the sake of making this example more
interesting, we modified the semantics of the instruction~\hs{abs} and
added~1~extra clock cycle in case of a negative argument by conditionally performing
\hs{delay 1} in the state transformer \hs{abs}.


\begin{minted}{haskell}
timingAnalysis = optimize Independent $ do
    t1 <- exists "t1"
    t2 <- exists "t2"
    p1 <- exists "p1"
    p2 <- exists "p2"
    constrain $ t1 .>= 0
            &&& t1 .<= toMilliSeconds (30 :: Year)
    constrain $ t2 .>= 0
            &&& t2 .<= toMilliSeconds (30 :: Year)
    constrain $ p1 .>= 0
            &&& p1 .<= toMilliWatts   ( 1 :: Watt)
    constrain $ p2 .>= 0
            &&& p2 .<= toMilliWatts   ( 1 :: Watt)
    let dataMemory = [t1, t2, p1, p2, 0, 100]
        finalState = simulate 100
                     (boot energyEstimateLowLevel
                           dataMemory)
    -- Specify independent optimisation goals:
    minimize "Best case"  (clock finalState)
    maximize "Worst case" (clock finalState)
\end{minted}


\noindent
The total delay of the program depends only on the sign of $t_1 - t_2$, thus
the best and worst cases differ only by one clock cycle. The worst case is
achieved when the difference is negative ($t_1 - t_2 = -2$), as shown below.
Z3 finishes in 0.5s.


% \begin{minipage}{0.45\textwidth}
\begin{minted}[frame=single]{text}
Objective "Best case":
Optimal model:
  t1        = 549755813888
  t2        =  17179869184
  p1        =            0
  p2        =            0
  Best case =           12
\end{minted}
% \end{minipage}
% \begin{minipage}{0.45\textwidth}
\begin{minted}[frame=single]{text}
Objective "Worst case":
Optimal model:
  t1         = 65535
  t2         = 65537
  p1         =     0
  p2         =     0
  Worst case =    13
\end{minted}
% \end{minipage}



