% For many spacecraft subsystems integrated circuits are required to perform control
% tasks or simple data processing. Typically, these integrated circuits are realised
% with FPGAs (Field Programmable Gate Arrays) due to their flexibility and lower
% costs compared to ASIC (Application-Specific Integrated Circuit) development \&
% fabrication. Since FPGAs can be used to implement arbitrary circuit functions
% including processor cores, it is possible to perform tasks both in hardware and
% in software. However, modern space-qualified FPGAs, which can withstand radiation
% in Earth orbit or deep space, have a limited amount of programmable resources.
% Therefore, it is often not feasible to implement a fully-fledged processor system
% in such an FPGA next to the mission-specific circuitry.

Many spacecraft subsystems rely on integrated circuits to perform control tasks
or simple data processing. Typically, these integrated circuits are realised
with FPGAs (Field Programmable Gate Arrays) due to their flexibility and lower
costs. Modern space-qualified FPGAs, which can withstand radiation in Earth
orbit or deep space, have a limited amount of programmable resources. Therefore,
it is often not feasible to implement a fully-fledged processor system in such
an FPGA next to the mission-specific circuitry.
The REDFIN instruction set was developed to address this issue and, more
specifically, to meet the following goals: (i) simple instruction set to
achieve a small hardware footprint, (ii) reduced complexity to support formal
verification of programs, and (iii) deterministic behaviour for real-time
applications.

\subsection{REDFIN instruction set and microarchitecture}


The instruction set architecture offers a configurable bit width for the
data path, ranging from 8 to 64 bits. Instruction words have a fixed width of 16
bits. The instruction set is based on a register-memory architecture, i.e.
instructions can fetch their operands from registers as well as directly from the
memory. This architecture favours a small register set, which minimises the hardware
footprint of the processing core. Furthermore, the number of instructions in a
program is typically smaller in comparison to traditional load/store architectures
where all operands have to be transferred to registers before any operations can
be performed. There are 47 instructions of the following types:

\begin{itemize}
\item{Load/store instructions for moving data between registers and memory, and
loading of immediate values.}
\item{Integer and fixed-point arithmetic operations.}
% In the latter
% case the number of fractional bits can be adjusted by a processor register.
\item{Bitwise logical and shift operations.}
\item{Control flow instructions and comparison operations.}
\item{Bus access instructions for read \& write operations on an AMBA AHB bus.}
\end{itemize}

The REDFIN processing core fetches instruction and data words from a small and fast
on-chip SRAM. This only allows for execution of simple programs, however, it also
eliminates the need to implement caches and thus removes a source of non-determinism of
conventional processors. Since computing performance is not one of the main goals, the
processor core is non-pipelined and therefore does not need to resolve data or control
hazards or perform any form of speculative execution. These properties greatly simplify
worst case execution time analysis.


\subsection{Requirements for formal verification}


Verification of \emph{functional correctness} of REDFIN programs, as defined by a
requirement specification, clearly is an essential task for the development of space
electronics. There are also important \emph{non-functional requirements}, such as
worst case execution time and energy consumption, which rely on the implementation
guarantees provided by the processing core. In order to reduce verification complexity,
the REDFIN core only allows to execute a single subroutine whose execution is triggered
by a higher-level controller in the system. The implementation also guarantees
that concurrent bus accesses to the processor registers or memory do not affect
the subroutine execution time. Furthermore, the processor does not implement
interrupt handling. All these measures are taken in order to provide real-time
subroutine execution guarantees and make the verification of non-functional
properties feasible within the presented verification framework.

Despite these restrictions the REDFIN core has already proven its effectiveness for
simple control tasks and arithmetic computations as part of an antenna pointing unit
for satellites. Nevertheless, verification can be difficult and time-consuming,
even for small and simple programs. Verification activities, following engineering
standards for space electronics, typically outweigh programming and design tasks by a
factor of two in terms of development hours. Usually verification is performed via
program execution on an instruction set simulator or a hardware model of the processor.
Manually deriving test cases from the specification is cumbersome and error-prone
and simulation times can become prohibitively long with a large number of tests that
are often needed to reach the desired functional and code coverage. Formal verification
methods can prove that a program satisfies certain properties for all possible
test cases and are therefore immensely valuable for completing the verification
with superior efficiency and quality.
