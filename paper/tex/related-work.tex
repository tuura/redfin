There is a vast body of literature available on the topic of formal verification,
including verification of hardware processing cores and low-level software programs.
Our work builds in a substantial way on a few known ideas that we will review in
this section. We thank the formal verification and programming languages
communities and hope that the formal semantics of the REDFIN processing core
will provide a new interesting benchmark for future studies.

We model the REDFIN microarchitecture using a~\emph{monadic state transformer
metalanguage} -- an idea with a long history.
\citet{fox2010trustworthy} formalise the Arm v7 instruction
set architecture in HOL4 and give a careful account to bit-accurate proofs of
the instruction decoder correctness. Later, \citet{kennedy2013coq}
formalised a subset of the x86 architecture in Coq, using monads for instruction
execution semantics and \textsf{do}-notation for assembly language embedding.
% Both these models are formalised in proof assistants, thus are powered by full
% dependent types, which allow the usage of mechanised program correctness proofs.
\citet{degenbaev2012formal} formally specified the \emph{complete} x86
instruction set -- a truly monumental effort! -- using a custom domain-specific
language that can be translated to a formal proof system. Arm's Architecture
Specification Language (ASL) has been developed for the same purpose to formalise the
Arm v8 instruction set~\cite{reid2016cav}. The SAIL language~\cite{SAIL-lang} has
been designed as a generic language for ISA specification and was used to
specify the semantics of ARMv8-a, RISC-v, and CHERI-MIPS.
Our specification approach is similar to these three works, but we operate on a
much smaller scale of the REDFIN core and focus on verifying whole programs.

Our metalanguage is embedded in Haskell and does not have a rigorous
formalisation, i.e. we cannot prove the correctness of the REDFIN semantics
itself, which is a common concern, e.g. see~\citet{reid2017oopsla}. Moreover, our
verification workflow mainly relies on \emph{automated} theorem proving, rather
than on \emph{interactive} one. This is motivated by the cost of precise proof
assistant formalisations in terms of human resources: automated techniques are
more CPU-intensive, but cause less ``human-scaling issues''~\cite{reid2016cav}.
Our goal was to create a framework that could be seamlessly
integrated into an existing spacecraft engineering workflow, therefore it needed
to have as much proof automation as possible. The automation is achieved by means
of \emph{symbolic program execution}. \citet{Currie2006} applied
symbolic execution with uninterpreted functions to prove equivalence of low-level
assembly programs. The framework we present allows not only proving the
equivalence of low-level programs, but also their compliance with higher-level
specifications written in a subset of Haskell.

% A lot of research work has been done on the design of \emph{typed assembly
% languages}, e.g. see~\cite{Haas:2017:BWU:3140587.3062363}\cite{Morrisett:1999:SFT:319301.319345}.
% The low-level REDFIN assembly is untyped, but the syntactic language of
% arithmetic expressions that we implemented on top of it does have a simple type
% system. In principle, the REDFIN assembly itself may benefit from a richer type
% system, especially one enforcing correct operation with relevant mission-specific
% units of measurement~\cite{Kennedy:1997:RPU:263699.263761}.

Finally, we would like to acknowledge the projects and talks
that provided an initial inspiration for this work: the `Monads to Machine
Code' compiler by \citet{diehl-monads-to-machines}, RISC"/V semantics
by~\citet{riscv-semantics}, the assembly monad by~\citet{asm-monad}, and
SMT-based program analysis by \citet{haskell-z3}.
