
% The presented approach has been implemented and is planned to be released at the
% conference (this paper is currently under review).

% In this section we discuss our design choices and achieved
% results, comparing them with the project's initial goals: (i)~providing a unified
% specification, testing, and formal verification framework that is (ii)~understandable
% and convenient to use by the REDFIN engineering team, and (iii)~allows the team
% to co-develop REDFIN software and hardware, by extending and modifying the
% default instruction semantics.

% \subsection{From hardware to untyped assembly to typed software}

The proposed approach covers two levels of organisation of computer systems: the
hardware microarchitecture (the state monad \hs{Redfin},~\S\ref{sec-transformer}),
and the instruction set architecture (the assembly monad
\hs{Script},~\S\ref{sec-verification}). These two levels are very different: the
former allows hardware engineers to precisely capture the program semantics (and
has proven useful in exposing underspecified behaviours), whereas the latter does
not have a direct access to the microarchitectural level and is used to
symbolically \emph{execute} the semantics, allowing software engineers to
\emph{observe} the results and \emph{reason} about program correctness.
By using Haskell as a metalanguage, we provided a purely syntactic implementation
of a higher-level abstraction on top of the REDFIN assembly~---~a statically-typed
language for arithmetic expressions. This demonstrates that the user
% of the verification framework
can implement their own domain-specific
extensions of the REDFIN assembly.

Although our current implementation is written in Haskell, the presented ideas
can be implemented in other languages. We chose Haskell for its built-in
support for polymorphic expressions, powerful \hs{do}-notation, and availability
of a mature symbolic manipulation library (SBV). We also have a prototype
implementation in Idris, a much younger language that features dependent
types and allows us to verify more sophisticated properties at the type
level~\cite{JFP:9060502}, however at the time of writing there is no equivalent
of the SBV library in Idris, which is a significant practical disadvantage.
Dependent Haskell~\cite{weirich2017dependent}, once implemented, will provide
a convenient alternative.

% \subsection{Uniform development, testing and verification environment}

The \hs{Script} monad was engineered to provide familiar assembly mnemonics and
directives (e.g. data and instruction labels), which allows engineers to start
using the framework for developing REDFIN programs even without prior experience
of Haskell development, hopefully increasing the uptake of the framework.

Thanks to symbolic simulation, we can uniformly handle both concrete and
symbolic values, thus testing becomes just a special case of formal verification,
allowing the engineers to reuse a common code base and infrastructure.
Testing yields trivial SMT problems that can be solved in sub-second time for
all programs of realistic sizes (typical REDFIN programs have hundreds of
instructions). Formal verification is more expensive: in our experiments, we
could handle programs comprising hundreds of instructions in 10-15 minutes, but
one can easily construct small programs that will grind any SMT solver to a halt:
for example, analysis of a single multiplication instruction \hs{mul} can take
half an hour if it is required to factor 64"/bit numbers (try factoring
4611686585363088391 with an SMT solver). In such cases, conservatively proving
some of the correctness properties at the type level can significantly increase
the productivity.

% Although proving properties about the hardware implementation is left for future
% work, the developed infrastructure provides a way to generate testsuites for the
% processing core from the formal semantics of REDFIN instructions. Furthermore,
% one can use the semantics to generate parts of the hardware
% implementation~\cite{reid2016cav} or synthesise efficient instruction
% subsets~\cite{mokhov2014synthesis}.
