
In this section we formally define the REDFIN microarchitecture and express the
semantics of the instruction set as an explicit and symbolic state transformer.


\subsection{The REDFIN microarchitecture state}


The main idea of the presented approach is to use an explicit state"/transformer
semantics of the REDFIN microarchitecture. The state space of the entire
processing core is a Cartesian product of state spaces of every component:

\begin{equation*}
\begin{split}
S=\{(r, m, ic, ir, p, f, c) : r \in R, m \in M, ic \in A, \\ir \in I,
p \in P, f \in F, c \in C\},
\end{split}
\end{equation*}

\noindent
where $R$ is the set of register bank configurations;
$M$ is the memory state space;
$A$~is the set of instruction addresses (the instruction counter $ic$ stores the
address of the current instruction);
$I$ is the set of instruction codes (the instruction register $ir$ stores the
code of the current instruction);
$P$ is the set of programs;
$F$ is the set of the flag register configurations; and
$C$ is the set of clock values.

Fig.~\ref{fig-types} shows the translation of the above into Haskell types. Note
that the types are not parameterised: recall that REDFIN is parameterised, e.g.
the data width can be chosen depending on mission requirements, whereas we use fixed
64-bit data path for the sake of simplicity. The chosen names are self-explanatory,
for example, the data type \hs{State} directly corresponds to the set of states~$S$.
We defined \hs{SymbolicValue} and \hs{SymbolicArray} type constructors on top of
Levent Erkok's symbolic verification library SBV~\cite{SBV}, which we use as the
SMT translation and verification frontend. \todo{In principle, any other SMT frontend
can be used, but to the best of our knowledge, SBV is the most mature SMT library
available for Haskell. \textbf{No, SBV here is crucial because it does all the actual symbolic execution}} We briefly overview all \hs{State} components below.

\begin{figure}[t]

\begin{minted}{haskell}
data State = State
  { registers           :: RegisterBank
  , memory              :: Memory
  , instructionCounter  :: InstructionAddress
  , instructionRegister :: InstructionCode
  , program             :: Program
  , flags               :: Flags
  , clock               :: Clock }
\end{minted}

\begin{minted}{haskell}
type Register           = SymbolicValue Word2
type Value              = SymbolicValue Int64
type RegisterBank       = SymbolicArray Word2 Int64
\end{minted}

\begin{minted}{haskell}
type MemoryAddress      = SymbolicValue Word8
type Memory             = SymbolicArray Word8 Int64
\end{minted}

\begin{minted}{haskell}
type InstructionAddress = SymbolicValue Word8
type InstructionCode    = SymbolicValue Word16
type Program            = SymbolicArray Word8 Word16
\end{minted}

\begin{minted}{haskell}
data Flag               = Condition | Overflow | ...
type Flags              = SymbolicArray Flag Bool
\end{minted}

\begin{minted}{haskell}
type Clock              = SymbolicValue Word64
\end{minted}

\caption{Basic types for encoding REDFIN microarchitecture in Haskell.\label{fig-types}}

\end{figure}


\subsubsection{Data values, registers and memory}
64-bit data values (\hs{Int64}) are stored in registers and memory. There are~4
registers (addressed by \hs{Word2}) and 256 memory locations (addressed by
\hs{Word8}). The register bank and memory are represented by symbolic arrays that can be
accessed via SBV's functions \hs{readArray} and \hs{writeArray}.


\subsubsection{Instructions and programs}
REDFIN instructions are represented by 16-bit \hs{InstructionCode}, whose 6
leading bits contain the instruction opcode, and the remaining 10 bits are
allocated for various instruction arguments. The \hs{Program} is a symbolic
array mapping 8-bit instruction addresses to instruction codes.


\subsubsection{Status flags and clock}
The microarchitecture stores execution status flags to support conditional
branching, track integer overflow, and terminate the program, as captured by the
data type \hs{Flag} (we omit a few other flags for brevity). The flag register is a
symbolic map from flags to Boolean values. The \hs{Clock} is a 64-bit counter
incremented on each clock cycle. Status flags and the clock are used for
diagnostic, formal verification and worst-case execution time analysis.


\subsection{Instruction and program semantics}

We can now define the formal semantics of REDFIN instructions and programs in terms
of a \emph{state transformer} $T : S \rightarrow S$, i.e. a function that maps
states to states. We distinguish between instructions and programs by using
Haskell's list notation, for example, $T_{\subhs{nop}}$ is the semantics of the
instruction $\hs{nop} \in I$, whereas $T_{\subhs{[}\subhs{nop}\subhs{]}}$ is the
semantics of the single-instruction program $\hs{[}\hs{nop}\hs{]} \in P$.

% \footnote{REDFIN does not have a dedicated \hs{nop}
% instruction, but one can use the semantically equivalent instruction \hs{jmpi 0}
% instead (i.e. jump to the next instruction).}

\textbf{Definition (program semantics):} The semantics of a program $p \in P$
is inductively defined as follows:


\begin{itemize}
    \item The semantics of the \emph{empty program} $\hs{[}\hs{]} \in P$ coincides with
    the semantics of the instruction \hs{nop} and is the identity state transformer:
    $T_{\subhs{[}\subhs{]}} = T_{\subhs{nop}} = \hs{id}$.
    
    \item The semantics of the \emph{single-instruction program} $\hs{[}\hs{i}\hs{]} \in P$
    is a composition of three state transformers: (i) fetching the instruction from
    the program memory (denoted by $T_\textit{fetch}$), (ii) incrementing the
    instruction counter ($T_\textit{inc}$), and (iii) the state transformer
    of the instruction itself ($T_{\subhs{i}}$), that is:
    
    \[
    \begin{array}{lcl}
    T_\textit{fetch} & = & (r, m, ic, ir, p, f, c) \mapsto (r, m, ic, p[ic], p, f, c + 1)\\
    T_\textit{inc} & = & (r, m, ic, ir, p, f, c) \mapsto (r, m, ic + 1, ir, p, f, c)\\
    T_{\subhs{[}\subhs{i}\subhs{]}} & = & T_{\subhs{i}} \circ T_\textit{inc} \circ T_\textit{fetch}.\\
    \end{array}
    \]
    
    \item The semantics of the \emph{composite program} $\hs{i}\hs{:}\hs{p} \in P$,
    where the operator~\hs{:}~prepends an instruction $\hs{i} \in I$ to a program
    $\hs{p} \in P$, is defined as $T_{\subhs{i}\subhs{:}\subhs{p}} = T_{\hs{p}} \circ T_{\subhs{[}\subhs{i}\subhs{]}}$.
    
\end{itemize}

\noindent
We represent state transformers in Haskell using the \emph{state monad}, a
classic approach to emulating mutable state in a purely functional programming
language~\cite{wadler1990comprehending}. We call our state monad~\hs{Redfin} and
define it\footnote{A generic version of this monad is available in standard module
\hs{Control.Monad.State}.} as follows:


\begin{minted}{haskell}
data Redfin a = Redfin
  { transform :: State -> (a, State) }
\end{minted}


\noindent
Every computation with the return type~\hs{Redfin}~\hs{a} yields a value of type~\hs{a}
and possibly alters the \hs{State} of the REDFIN microarchitecture. As an example,
below we express the state transformer $T_\textit{inc}$ using the \hs{Redfin} monad.


\begin{minted}{haskell}
incrementInstructionCounter :: Redfin ()
incrementInstructionCounter =
  Redfin $ \current -> ((), next)
  where next = current { instructionCounter =
          instructionCounter current + 1 }
\end{minted}


\noindent
In words, the state transformer looks up the value of the \hs{instructionCounter}
in the \hs{current} state and replaces it in the \hs{next} state with the
incremented~value. The type \hs{Redfin}~\hs{()} indicates that the computation does not
produce any value as part of the state transformation. Such computations directly
correspond to REDFIN programs and can be composed using the operator \hs{>>}.
For example, \hs{fetchInstruction}\\\hs{>>}~\hs{incrementInstructionCounter} is
the state transformer $T_\textit{inc} \circ T_\textit{fetch}$ assuming that
\hs{fetchInstruction} corresponds to $T_\textit{fetch}$. We can also use
Haskell's powerful \hs{do}-notation to compose computations:


\begin{minted}{haskell}
readInstructionRegister :: Redfin InstructionCode
readInstructionRegister =
  Redfin $ \s -> (instructionRegister s, s)
\end{minted}

\begin{minted}{haskell}
executeInstruction :: Redfin ()
executeInstruction = do
  fetchInstruction
  incrementInstructionCounter
  instructionCode <- readInstructionRegister
  decodeAndExecute instructionCode
\end{minted}

\noindent
Here \hs{readInstructionRegister} extracts the instruction code from the current
state \emph{without modifying it}. This function is used in \hs{executeInstruction},
which defines the semantics of the REDFIN execution cycle. We omit the definition of
\hs{decodeAndExecute} for brevity: it is a case analysis of 47 opcodes that returns
the matching instruction. We introduce several interesting instructions below.


\subsubsection{Halting the processor}
If the~\hs{halt} instruction is encountered, the processor sets the flag~\hs{Halt},
thereby stopping the execution of the current subroutine until a new one is
started by a higher-level system controller that resets \hs{Halt}. The auxiliary
function \hs{writeFlag} is used to do the actual flag modification.


\begin{minted}{haskell}
halt :: Redfin ()
halt = writeFlag Halt true
\end{minted}

\begin{minted}{haskell}
writeFlag :: Flag -> SymbolicValue Bool -> Redfin ()
writeFlag flag value = Redfin $ \s -> ((), s')
  where s' = s { flags = writeArray (flags s)
                         (flagId flag) value }
\end{minted}



\subsubsection{Arithmetics}
As a more involved example, consider the semantics of the instruction \hs{abs}.
It reads a register and writes back the absolute value of its
contents\footnote{We use \hs{Prelude.abs} to distinguish between the instruction
and the function from the standard library \hs{Prelude}; \hs{fmap} applies
\hs{Prelude.abs} to the result of \hs{readRegister}.}.
The semantics accounts for the potential integer overflow that leads to the
\emph{negative resulting value} when the input is $-2^{63}$ (REDFIN
uses the two's complement signed number representation). The overflow is flagged
by setting~\hs{Overflow}.


\begin{minted}{haskell}
abs :: Register -> Redfin ()
abs rX = do
    state  <- readState
    result <- fmap Prelude.abs (readRegister rX)
    let (_, overflowState) =
      transform (writeFlag Overflow true) state
    writeState $ ite (result .< 0)
                     overflowState state
    writeRegister rX result
\end{minted}


\noindent
Here, SBV's symbolic \emph{if-then-else} operation~\hs{ite} is used to \emph{merge}
two possible next states, one of which has the \hs{Overflow} flag set. We use
auxiliary functions \hs{readRegister}, \hs{writeRegister}, \hs{readState} and
\hs{writeState} --- simple state transformers defined similarly to
\hs{readInstructionRegister} and \hs{writeFlag}.


\subsubsection{Conditional branching}
As an example of a control flow instruction, consider the conditional branch
\hs{jmpi_ct}, which tests the~\hs{Condition} flag, and adds the
provided offset to the instruction counter if the flag is set.


\begin{minted}{haskell}
jmpi_ct :: SymbolicValue Int8 -> Redfin ()
jmpi_ct offset = do
  ic <- readInstructionCounter
  condition <- readFlag Condition
  let ic' = ite condition (ic + offset) ic
  writeInstructionCounter ic'
\end{minted}


\noindent
After working through the above examples, it is worth noting that we use our
Haskell encoding of the state transformer as a~\emph{metalanguage}. We are
operating the REDFIN core as a puppet master, using external
meta-notions of addition,
comparison and let"/binding. From the processor's
point of view, we have infinite memory and act instantly, which gives us unlimited
modelling power. For example, we can run a simulation of the processor environment
in an external tool and feed its result to \hs{writeRegister} as if it was
obtained in a single clock cycle.

\subsection{Symbolic simulation}

Having defined the semantics of REDFIN instructions and programs, we can
implement symbolic simulation of the processor:

\begin{minted}{haskell}
simulate :: Int -> State -> State
simulate steps s
    | steps <= 0 = s
    | otherwise  =
        ite halted s (simulate (steps - 1) nextS)
  where
    halted = readArray (flags s) (flagId Halt)
    nextS  = snd $ transform executeInstruction s
\end{minted}

\noindent
The function takes a number of simulation steps $N$ and an initial symbolic
state of the processor as input, and executes $N$ instructions using the
previously defined \hs{executeInstruction} function. In each \hs{state} we need
to merge two possible futures depending on the value of the \hs{Halt} flag:
(i) continue the simulation starting from the \hs{nextState} if the flag is not
set, and (ii) remain in the current \hs{state} if the flag is set, since in
this case the processor must remain idle.

%TODO: What about program synthesis?
Symbolic simulation is very powerful. It allows us to formally verify properties
of REDFIN programs by fixing some parts of the state to constant values (for
example, the program code), and then checking assertions on the resulting values of
the symbolic part of the state. This will be discussed in the next
section~\S\ref{sec-verification}.

