\section{Formal verification in presence of unbounded loops\label{sec-motor-control}}

In the last section we have demonstrated the framework's capabilities for
formal verification on an example of the energy estimation program.
Many programs targeting REDFIN share the distinctive feature of
this example and have statically predictable upper \emph{bound on execution time}
thanks to the fact that their termination does not
depend on the input data. However, some control programs may have loops
which are guarded by termination conditions that involve computation
considering the input parameters of the program, thus making the loop
\emph{unbounded}.

\todo{Where to put the discussion of symbolic termination?}

The presence of unbounded loops makes whole-program
verification inapplicable in the general case~\cite{some-paper-on-symexec},
since the state space of the program becomes infinite.

In this section we apply the developed verification framework to a larger
example of a stepper motor control program and make an accent on how
the presence of an unbounded loop may be mitigated and which properties
of the program can be verified with symbolic execution.

\subsection{Stepper motor control program}

High-level pseudo code

Figure with two graphs

LaTeX specification

Diagram of states of the program maybe

How to cut the loop out