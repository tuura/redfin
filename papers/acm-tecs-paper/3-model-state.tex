\section{Modelling REDFIN in Haskell}\label{sec-transformer}

In this section we formally define the REDFIN microarchitecture and express the
semantics of the instruction set as an explicit and symbolic state transformer.

\begin{figure}[t]
\begin{minted}[xleftmargin=10pt]{haskell}
data State = State
  { registers           :: RegisterBank
  , memory              :: Memory
  , instructionCounter  :: InstructionAddress
  , instructionRegister :: InstructionCode
  , program             :: Program
  , flags               :: Flags
  , clock               :: Clock }

type Register           = SymbolicValue Word2
type Value              = SymbolicValue Int64
type RegisterBank       = SymbolicArray Word2 Int64
type MemoryAddress      = SymbolicValue Word8
type Memory             = SymbolicArray Word8 Int64

type InstructionAddress = SymbolicValue Word8
type InstructionCode    = SymbolicValue Word16
type Program            = SymbolicArray Word8 Word16

data Flag               = Condition@\,@|@\,@Overflow@\,@|@\,@Halt@\,@...
type Flags              = SymbolicArray Flag Bool
type Clock              = SymbolicValue Word64
\end{minted}
\caption{Basic types for modelling REDFIN.\label{fig-types}}
\end{figure}

\subsection{The REDFIN Microarchitecture State}

The main idea of our approach is to use an explicit state transformer
semantics of the REDFIN microarchitecture. The \hs{State} of the entire
processing core is a product of states of every component, see
Fig.~\ref{fig-types}. We define \hs{SymbolicValue} and \hs{SymbolicArray} on top
of the SBV library~\cite{SBV} that we use as a frontend for SMT translation and
verification.

% \begin{equation*}
% \begin{split}
% S=\{(r, m, ic, ir, p, f, c) : r \in R, m \in M, ic \in A, \\ir \in I,
% p \in P, f \in F, c \in C\},
% \end{split}
% \end{equation*}

% \noindent
% where $R$ is the set of register bank configurations;
% $M$ is the memory state space;
% $A$~is the set of instruction addresses (the instruction counter $ic$ stores the
% address of the current instruction);
% $I$ is the set of instruction codes (the instruction register $ir$ stores the
% code of the current instruction);
% $P$ is the set of programs;
% $F$ is the set of the flag register configurations; and
% $C$ is the set of clock values.

% Fig.~\ref{fig-types} shows the translation of the above into Haskell types. Note
% that the types are not parameterised: recall that REDFIN is parameterised, e.g.
% the data width can be chosen depending on mission requirements, whereas we use fixed
% 64-bit data path for the sake of simplicity. The chosen names are self-explanatory,
% for example, the data type \hs{State} directly corresponds to the set of states~$S$.

% \todo{In principle, any other SMT frontend
% can be used, but to the best of our knowledge, SBV is the most mature SMT library
% available for Haskell. \textbf{No, SBV here is crucial because it does all the actual symbolic execution}} We briefly overview all \hs{State} components below.

% \subsubsection{Data values, registers and memory}

There are~4 registers (addressed by \hs{Word2}) and 256 memory cells (addressed
by \hs{Word8}) that store 64-bit values (\hs{Int64}). The register bank and
memory are represented by symbolic arrays
that can be accessed via SBV's functions \hs{readArray} and \hs{writeArray}.
REDFIN uses 16-bit \hs{InstructionCode}s, whose 6 leading bits contain the
opcode, and the remaining 10 bits hold instruction arguments. The \hs{Program}
maps 8-bit instruction addresses to instruction codes.

The microarchitecture status \hs{Flags} support conditional branching, track
integer overflow, and terminate the program (we omit a few other flags for
brevity).
The \hs{Clock} is a 64-bit counter
incremented on each clock cycle. Status flags and the clock are used for
diagnostics, formal verification, and worst-case execution time analysis.

\subsection{Instruction and Program Semantics}

We can now define the formal semantics of REDFIN instructions and programs as a
\emph{state transformer} $T : S \rightarrow S$, i.e. a function that maps
states to states. We distinguish instructions and programs by using
Haskell's list notation, e.g. $T_{\subhs{nop}}$ is the semantics of the
instruction $\hs{nop} \in I$, whereas $T_{\subhs{[}\subhs{nop}\subhs{]}}$ is the
semantics of the single-instruction program $\hs{[}\hs{nop}\hs{]} \in P$.
\footnote{REDFIN does not have a dedicated \hs{nop} instruction, but it can be
expressed as a jump to the next instruction, i.e.~\hs{jmpi 0}.}

\vspace{-1.5mm}
\noindent\hrulefill~\\
\vspace{-3.5mm}

\noindent
\textbf{Definition (program semantics):} The semantics of a program $p \in P$
is inductively defined as follows:

% \begin{itemize}
% \item
    The semantics of the \emph{empty program} $\hs{[}\hs{]} \in P$ coincides with
    the semantics of the instruction \hs{nop} and is the identity state transformer:
    $T_{\subhs{[}\subhs{]}} = T_{\subhs{nop}} = \hs{id}$.

    % \item
    The semantics of a \emph{single-instruction program} $\hs{[}\hs{i}\hs{]} \in P$
    is a composition of (i) fetching the instruction from
    the program memory~$T_\textit{fetch}$, (ii) incrementing the
    instruction counter~$T_\textit{inc}$, and (iii) the state transformer
    of the instruction itself~$T_{\subhs{i}}$; or,~using the order of state
    components from Fig.~\ref{fig-types}:
    \vspace{-1mm}
    \[
    \begin{array}{lcl}
    T_\textit{fetch} & = & (r, m, ic, ir, p, f, c) \mapsto (r, m, ic, p[ic], p, f, c + 1)\\
    T_\textit{inc} & = & (r, m, ic, ir, p, f, c) \mapsto (r, m, ic + 1, ir, p, f, c)\\
    T_{\subhs{[}\subhs{i}\subhs{]}} & = & T_{\subhs{i}} \circ T_\textit{inc} \circ T_\textit{fetch}\\
    \end{array}
    \]

    % \item
    % \vspace{-1mm}
    \noindent
    The semantics of a \emph{composite program} $\hs{i}\hs{:}\hs{p} \in P$,
    where the operator~\hs{:}~prepends an instruction $\hs{i} \in I$ to a program
    $\hs{p} \in P$, is defined as $T_{\subhs{i}\subhs{:}\subhs{p}} = T_{\hs{p}} \circ T_{\subhs{[}\subhs{i}\subhs{]}}$.

% \end{itemize}

\vspace{-1mm}
\noindent\hrulefill~\\
\vspace{-3mm}

% \noindent
We represent state transformers in Haskell using the \emph{state monad}, a
classic approach to emulating mutable state in a purely functional programming
language~\cite{wadler1990comprehending}. We call our state monad~\hs{Redfin} and
define it as follows\footnote{A generic version of this monad is available in
the standard module \hs{Control.Monad.State}.}:

\vspace{1mm}
\begin{minted}[xleftmargin=10pt]{haskell}
data Redfin a = Redfin { transform :: State -> (a, State) }
\end{minted}
\vspace{1mm}

\noindent
A computation of type~\hs{Redfin}~\hs{a} yields a value of type~\hs{a}~and
possibly alters the \hs{State} of the REDFIN microarchitecture.
The type \hs{Redfin}~\hs{()} describes a computation that does not produce any
value as part of the state transformation; such computations directly correspond
to state transformers.

\noindent
For example, here is the state transformer $T_\textit{inc}$:

\vspace{1mm}
\begin{minted}[xleftmargin=10pt]{haskell}
incrementInstructionCounter :: Redfin ()
incrementInstructionCounter = Redfin $ \current -> ((), next)
  where
    next = current { instructionCounter = instructionCounter current + 1 }
\end{minted}
\vspace{1mm}

\noindent
In words, the state transformer looks up the value of the \hs{instructionCounter}
in the \hs{current} state and replaces it in the \hs{next} state with the
incremented value. We can compose such primitive computations into more complex
state transformers using Haskell's \hs{do}-notation:

\vspace{1mm}
\begin{minted}[xleftmargin=10pt]{haskell}
readInstructionRegister :: Redfin InstructionCode
readInstructionRegister = Redfin $ \s -> (instructionRegister s, s)

executeInstruction :: Redfin ()
executeInstruction = do
    fetchInstruction
    incrementInstructionCounter
    instructionCode <- readInstructionRegister
    decodeAndExecute instructionCode
\end{minted}
\vspace{1mm}

\noindent
Here \hs{readInstructionRegister} reads the instruction code from the current
state \emph{without modifying it}, and is subsequently used in \hs{executeInstruction},
which defines the semantics of the REDFIN execution cycle. We omit definitions of
\hs{fetchInstruction} and \hs{decodeAndExecute} for brevity. The latter is a
case analysis of 47 opcodes that returns the matching instruction. We discuss
several instructions below.

\subsubsection{Halting the Processor}
The instruction~\hs{halt} sets the flag~\hs{Halt}, which stops the execution of
the current subroutine until a new one is started by a higher-level system
controller that resets~\hs{Halt}.

\vspace{1mm}
\begin{minted}[xleftmargin=10pt]{haskell}
halt :: Redfin ()
halt = writeFlag Halt true
\end{minted}
\vspace{1mm}

\noindent
The auxiliary function \hs{writeFlag} modifies the flag:

\vspace{1mm}
\begin{minted}[xleftmargin=10pt]{haskell}
writeFlag :: Flag -> SymbolicValue Bool -> Redfin ()
writeFlag flag value = Redfin $ \s -> ((), s')
  where
    s' = s { flags = writeArray (flags s) (flagId flag) value }
\end{minted}
\vspace{1mm}

\noindent
In the rest of the paper we will use auxiliary functions \hs{readRegister},
\hs{writeRegister}, \hs{readState}, etc.; they are simple state transformers
defined similarly to \hs{writeFlag}.

\subsubsection{Arithmetics}
The instruction \hs{abs} is more involved:
it reads a register and writes back the absolute value of its contents.
The semantics accounts for the potential integer overflow that leads to the
\emph{negative resulting value} when the input is $-2^{63}$ (REDFIN uses the
common two's complement signed number representation). The overflow is flagged
by setting~\hs{Overflow}. We use SBV's symbolic \emph{if-then-else}
operation~\hs{ite} to \emph{merge} two symbolic values~---~in this case two
possible next states, one of which is a state with the \hs{Overflow} flag set:

\vspace{1mm}
\begin{minted}[xleftmargin=10pt]{haskell}
abs :: Register -> Redfin ()
abs reg = do
    state  <- readState
    result <- fmap Prelude.abs (readRegister reg)
    let (_, state') = transform (writeFlag Overflow true) state
    writeState $ ite (result .< 0) state' state
    writeRegister reg result
\end{minted}
\vspace{1mm}

\subsubsection{Conditional Branching}
As an example of a control flow instruction consider \hs{jmpi_ct}, which
tests the~\hs{Condition} flag, and adds the provided \hs{offset} to the
instruction counter if the flag is set.

\vspace{1mm}
\begin{minted}[xleftmargin=10pt]{haskell}
jmpi_ct :: SymbolicValue Int8 -> Redfin ()
jmpi_ct offset = do
    ic <- readInstructionCounter
    condition <- readFlag Condition
    let ic' = ite condition (ic + offset) ic
    writeInstructionCounter ic'
\end{minted}
\vspace{1mm}

\noindent
% After working through the above examples, it is worth noting that
We use our Haskell encoding of the state transformer as a~\emph{metalanguage}:
we operate the REDFIN core as a puppet master, using external meta-notions of
addition, comparison and let-binding. From the processor's
point of view, we have infinite memory and act instantly, which gives us unlimited
modelling power. For example, we can simulate the processor environment
in an external tool and feed its result to \hs{writeRegister} as if it was
obtained in one clock cycle.

\subsection{Symbolic simulation}

Having defined the semantics of REDFIN programs, we can perform
\emph{symbolic processor simulation}.

There are many flavours of symbolic execution~\cite{SurveySymExec-CSUR18}
and there is no single answer to the question of which one is the best.
The choice of the symbolic execution technique depends heavily on the
verification scenarios. The verification framework for REDFIN is designed
to deal with two main classes of programs: (i) arithmetic calculations that
are statically provable to be terminating and (ii) control programs with
unbounded loops whose termination depend on dynamic input data. We approach
verification of the former using~\emph{symbolic execution with merging} that
allows for whole-program verification by translating the program and the
verification condition into a single SMT formula. However, unconditional merging
does not work for non-terminating programs; thus
we employ the traditional approach to symbolic execution that generates from programs
the tree-shaped traces which could be split into separate execution paths for path-by-path
analysis and verification. The verification framework thus works in two
modes:~\emph{merging} and~\emph{branching}. See~\S\ref{sec-verification} for
a verification example using the merging mode and~\S\ref{sec-motor-control} for
an approach to verify a non-terminating program in branching mode.

\subsubsection{Merging Mode}

The function \hs{simulate} takes a number of simulation steps~$N$ and an initial
symbolic \hs{State} as input, and runs \hs{executeInstruction} defined above~$N$
times. In each \hs{state} we merge two possible futures: (i) if the \hs{Halt}
flag is set, we continue the simulation from the~\hs{next}~state, (ii) otherwise
we remain in the current \hs{state}, since in this case the processor must
remain idle.

\vspace{1mm}
\begin{minted}[xleftmargin=10pt]{haskell}
simulate :: Int -> State -> State
simulate steps s =
    if steps <= 0 then s else ite halted s (simulate (steps - 1) next)
  where
    halted = readArray (flags s) (flagId Halt)
    next   = snd (transform executeInstruction s)
\end{minted}
\vspace{1mm}

\noindent
The function~\hs{ite} here performs symbolic merging of two possible states:
where the halted flag is set and not. The underlying semantics of instructions
will also perform merging on encounter of every branching instruction.

\subsubsection{Branching Mode}

In branching mode, on the contrary, we do not perform any merging at all and
generate a rose tree shaped\footnote{As defined in the standard
module~\cmd{Data.Tree} of Haskell's~\cmd{containers} package.} symbolic
execution trace where every node stores an integer identifier, a program state
and its associated path constraints. We take advantage of the fact that the
semantics of the~\hs{halt} instruction updates the~\hs{Halt} flag with concrete
boolean value~\hs{True} and thus we can use the native Haskell conditional
operator to check if the current execution path has terminated and return a leaf
node in this case. Otherwise, we generate~\hs{futures} --- a list of successor
states, recursively calling the~\hs{simulate} function with every child as the
initial state and attaching the results as leaves to the current node.

\vspace{1mm}
\begin{minted}[xleftmargin=10pt]{haskell}
simulate :: Int -> State -> Tree (Node Int State)
simulate steps s = getTrace $ do
    modify (+1)
    n <- get
    let halted = readArray (flags s) (flagId Halt)
    if steps <= 0 || halted
    then pure (mkTrace (Node n s) [])
    else do futures <- traverse (simulate (steps - 1)) (executeInstruction s)
            pure $ mkTrace (Node n s) futures
\end{minted}
\vspace{1mm}

\noindent
Symbolic simulation is very powerful. It allows us to formally verify properties
of REDFIN programs by fixing some parts of the state to constant values (e.g.,
the program), and then making assertions on the resulting values of
the symbolic part of the state, as demonstrated in the
sections~\S\ref{sec-verification} and~\S\ref{sec-motor-control}.
