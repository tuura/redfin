\section{Discussion\label{sec-discussion}}
% The presented approach has been implemented and is planned to be released at the
% conference (this paper is currently under review).

% In this section we discuss our design choices and achieved
% results, comparing them with the project's initial goals: (i)~providing a unified
% specification, testing, and formal verification framework that is (ii)~understandable
% and convenient to use by the REDFIN engineering team, and (iii)~allows the team
% to co-develop REDFIN software and hardware, by extending and modifying the
% default instruction semantics.

% \subsection{From hardware to untyped assembly to typed software}

% The proposed approach covers two levels of organisation of computer systems: the
% hardware microarchitecture (the state monad \hs{Redfin},~\S\ref{sec-transformer}),
% and the instruction set architecture (the assembly monad
% \hs{Script},~\S\ref{sec-verification}). These two levels are very different: the
% former allows hardware engineers to precisely capture the semantics of
% instructions (and has proven useful in exposing underspecified behaviours),
% whereas the latter does not have a direct access to the microarchitectural level
% and is used to symbolically \emph{execute} the semantics, allowing software
% engineers to \emph{observe} the results and \emph{reason} about program
% correctness.

As the previous section~\S\ref{sec-verification} demonstrates, the presented
approach provides a unified specification, testing, and formal verification
framework. It allows the REDFIN engineering team to co-develop REDFIN software
and hardware, by extending and modifying the default instruction semantics.
By using Haskell as a metalanguage, one can implement higher-level languages on
top of the REDFIN assembly, such as our simple statically-typed language for
arithmetic expressions.

Static typing, polymorphism, \textsf{do}-notation, and availability of a mature
symbolic manipulation library~(SBV) were the key factors for choosing Haskell
for this project. We also have a prototype implementation in the
dependently-typed language Idris~\cite{JFP:9060502} that allows us to verify
more sophisticated properties at the type level, however at the time of writing
there is no equivalent of the SBV library in Idris, which is a significant
practical disadvantage.
% Dependent Haskell~\cite{weirich2017dependent} is a
% pos alternative.

% \subsection{Uniform development, testing and verification environment}

The \hs{Script} monad was engineered to provide familiar assembly mnemonics and
directives (e.g. labels), which allows engineers to start
using the framework for developing REDFIN programs even without prior Haskell
experience, hopefully increasing the uptake of the framework.

Thanks to symbolic simulation, we can uniformly handle both concrete and
symbolic values, reusing the same code base and infrastructure for testing and
formal verification.
Testing yields trivial SMT problems that can be solved in sub-second time.
Formal verification is more expensive: in our experiments,
programs of realistic sizes (typical REDFIN programs have hundreds of
instructions) required 10-15 minutes, but one can
easily construct tiny programs that will grind any SMT solver to a halt:
for example, analysis of a single multiplication instruction can take half an
hour if it is required to factor 64-bit numbers --- try to factor
4611686585363088391 with an SMT solver! In such cases, conservatively proving
some of the correctness properties at the type level can significantly increase
the productivity. As a microbenchmark, we verified the correctness of an array
summation program, reporting the number of SMT clauses and Z3 runtime for
low-level (LL) and high-level (HL) programs:

\noindent
{\smaller\begin{tabular}{l||c|c|c|c|c}
\hline
Benchmark        & Array & Clauses & Clauses & Time    & Time    \\
                 & size  & LL      & HL      & LL      & HL      \\\hline
Overflow:        & 9     & 286     & 260     & 1.482s  & 0.443s  \\
values not       & 12    & 492     & 453     & 3.604s  & 1.365s  \\
constrained      & 15    & 740     & 688     & 49.969s & 7.362s  \\
                 & 18    & 1030    & 965     & 76.757s & 88.458s \\\hline
No overflow:     & 9     & 318     & 292     & 0.467s  & 0.119s  \\
all values       & 12    & 549     & 510     & 0.739s  & 0.682s  \\
in $[1,1000]$    & 15    & 828     & 776     & 1.839s  & 6.408s  \\
                 & 18    & 1155    & 1090    & 9.944s  & 72.880s \\\hline
Equivalence      & 9     & 258     & 261     & 0.039s  & 0.097s  \\
to the \hs{sum}  & 12    & 495     & 459     & 0.053s  & 0.647s  \\
function         & 15    & 708     & 702     & 0.311s  & 7.322s  \\
                 & 18    & 1005    & 990     & 2.633s  & 71.896s \\\hline
\end{tabular}
}

% {\smaller
% \begin{tabular}{c||c|c|c|c|c}
% Array size & 6 & 9 & 12 & 15 & 18 \\\hline
% Low-level & 0.022s & 0.039s & 0.053s & 0.311s & 2.633s \\
% High-level & 0.040s & 0.097s & 0.647s & 7.322s & 71.896s \\
% \end{tabular}
% }

% Although proving properties about the hardware implementation is left for future
% work, the developed infrastructure provides a way to generate testsuites for the
% processing core from the formal semantics of REDFIN instructions. Furthermore,
% one can use the semantics to generate parts of the hardware
% implementation~\cite{reid2016cav} or synthesise efficient instruction
% subsets~\cite{mokhov2014synthesis}.

% \begin{table*}[t]
% \smaller
% \centering
% \begin{tabular}{l||c|c|c|c|c}
% \hline
% Benchmark        & Array & Formula & Formula & Time    & Time    \\
%                  & size  & size LL & size HL & LL      & HL      \\\hline
% Positive         & 9     & 286     & 260     & 1.482s  & 0.443s  \\
% overflow         & 12    & 492     & 453     & 3.604s  & 1.365s  \\
% check            & 15    & 740     & 688     & 49.969s & 7.362s  \\
%                  & 18    & 1030    & 965     & 76.757s & 88.458s \\\hline\hline
% Negative         & 9     & 318     & 292     & 0.467s  & 0.119s  \\
% overflow         & 12    & 549     & 510     & 0.739s  & 0.682s  \\
% check            & 15    & 828     & 776     & 1.839s  & 6.408s  \\
%                  & 18    & 1155    & 1090    & 9.944s  & 72.880s \\\hline\hline
% Equivalent       & 9     & 258     & 261     & 0.039s  & 0.097s  \\
% to the \hs{sum}  & 12    & 495     & 459     & 0.053s  & 0.647s  \\
% function         & 15    & 708     & 702     & 0.311s  & 7.322s  \\
%                  & 18    & 1005    & 990     & 2.633s  & 71.896s \\\hline
% \end{tabular}
% \caption{Verification of low-level and high-level array summation procedures
% \label{tab-sum-benchmark}}
% \caption*{\\$^{(*)}$assuming the summands are in range $[1, 1000]$.
% $^{(**)}$procedure is functionally equivalent to the Haskell's \hs{sum} function.}
% \end{table*}

% Benchmark        & Array & Formula & Formula & Time    & Time    \\
%                  & size  & size LL & size HL & LL      & HL      \\\hline
% Positive         & 6     & 122     & 109     & 0.503s  & 0.288s  \\
% overflow         & 9     & 286     & 260     & 1.482s  & 0.443s  \\
% check            & 12    & 492     & 453     & 3.604s  & 1.365s  \\
%                  & 15    & 740     & 688     & 49.969s & 7.362s  \\
%                  & 18    & 1030    & 965     & 76.757s & 88.458s \\\hline\hline
% Negative         & 6     & 135     & 122     & 0.171s  & 0.070s  \\
% overflow         & 9     & 318     & 292     & 0.467s  & 0.119s  \\
% check            & 12    & 549     & 510     & 0.739s  & 0.682s  \\
%                  & 15    & 828     & 776     & 1.839s  & 6.408s  \\
%                  & 18    & 1155    & 1090    & 9.944s  & 72.880s \\\hline\hline
% Equivalent       & 6     & 105     & 108     & 0.022s  & 0.040s  \\
% to the \hs{sum}  & 9     & 258     & 261     & 0.039s  & 0.097s  \\
% function         & 12    & 495     & 459     & 0.053s  & 0.647s  \\
%                  & 15    & 708     & 702     & 0.311s  & 7.322s  \\
%                  & 18    & 1005    & 990     & 2.633s  & 71.896s \\\hline
